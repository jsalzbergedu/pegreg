%% -*- TeX-command-extra-options: "-shell-escape" ; eval: (LaTeX-math-mode)  ; eval: (TeX-engine-set 'luatex)-*-
\documentclass[10pt]{article}
\usepackage[margin=1in]{geometry}
\author {Jacob Salzberg}
\date{\today}
\title{Potential Fixedpoint Semantics for PEGREG}

\usepackage[ligature, inference]{semantic}
\usepackage{galois}
\usepackage{array, longtable}
\usepackage{extarrows}
\usepackage{fontspec}
\usepackage{lilyglyphs}
\usepackage{stmaryrd}
\usepackage{amsmath}
\usepackage{amssymb}
\usepackage{bbm}
\usepackage{beramono}
\usepackage{unicode-math}
\usepackage{xcolor}
\usepackage{xparse}
\setmathfont{Libertinus Math}
\DeclareMathSizes{10}{9}{7}{5}
\DeclareMathAlphabet{\mathbbmsl}{U}{bbm}{m}{sl}
\newcommand{\emptytrace}{\ni}
\newcommand{\traceconcat}{\mathbin{\textcolor[HTML]{0a3d62}{\tiny{\fermata}}}}
\newcommand{\environments}{\textcolor[HTML]{b71540}{\mathbbm{Ev}}}
\newcommand{\pgmset}{{\mathbbm{P}}}
\newcommand{\pgmcmp}{\mathbbm{Pc}}
\newcommand{\pgm}{\mbox{P}}
\newcommand{\stmt}{\textcolor[HTML]{0a3d62}{\textsf{\mbox{S}}}}
\newcommand{\semantic}{{\mathcal{S}}}
\newcommand{\stmtlist}{\mathbin{\textcolor[HTML]{0a3d62}{\textsf{\mbox{Sl}}}}}
\newcommand{\emptysl}{\mathbin{\textcolor[HTML]{0a3d62}{\epsilon}}}
\newcommand{\stmtlistset}{\mathit{\mathbbm{Sl}}}
\newcommand{\stmtset}{\mathit{\textcolor[HTML]{0a3d62}{\mathbbm{S}}}}
\newcommand{\aexpr}{\textcolor[HTML]{b71540}{\mbox{\texttt{A}}}}
\newcommand{\aeval}{\mathcal{A}}
\newcommand{\beval}{\mathcal{B}}
\newcommand{\traces}{\mathscr{T}}
\newcommand{\Fp}{\mathscr{F}}
\newcommand{\bexpr}{\textcolor[HTML]{b71540}{\mbox{\texttt{B}}}}
\newcommand{\true}{\textcolor[HTML]{000000}{\textsf{\mbox{tt}}}}
\newcommand{\false}{\textcolor[HTML]{000000}{\textsf{\mbox{ff}}}}
\newcommand{\lbl}{{\textcolor[HTML]{3c6382}{ℓ}}}
\newcommand{\synif}[1]{{\kern-0.20em}\mathbin{\mbox{\texttt{\textbf{\textcolor[HTML]{0a3d62}{if}}}}}#1\mathbin{}}
\newcommand{\synassign}{\mathbin{\mbox{\texttt{\textbf{\textcolor[HTML]{0a3d62}{=}}}}}}
\newcommand{\synskip}{\mathbin{\mbox{\texttt{\textbf{\textcolor[HTML]{0a3d62}{;}}}}}}
\newcommand{\synelse}{\mathbin{\mbox{\texttt{\textbf{ \textcolor[HTML]{0a3d62}{else} }}}}}
\newcommand{\synwhile}[1]{{\kern-0.20em}\mathbin{\mbox{\texttt{\textbf{\textcolor[HTML]{0a3d62}{while}}}}}#1\mathbin{}}
\newcommand{\synbreak}{\mbox{\texttt{\textbf{\textcolor[HTML]{0a3d62}{break;}}}}}
\newcommand{\synvar}{{\mbox{{\texttt{\textbf{\textcolor[HTML]{b71540}{x}}}}}}}
\newcommand{\variables}{\textcolor[HTML]{b71540}{\mathbbmsl{V}}}
\newcommand{\tracevalues}{\textcolor[HTML]{000000}{\mathbbm{V}}}
\newcommand{\tracevalue}{\mbox{\textit{v}}}
\newcommand{\traceaction}{\mbox{{\tt a}}}
\newcommand{\traceactions}{\textcolor[HTML]{3c6382}{\mathbbmsl{A}}}
\newcommand{\trace}{\mathbbm{T}}
\newcommand{\tracearrow}[1]{{\textcolor[HTML]{3c6382}{\xrightarrow{\textcolor[HTML]{000000}{#1}}}}{}}
\newcommand{\nand}{\mbox{ nand }}
\newcommand{\pcat}{\textcolor[HTML]{3c6382}{\textsf{\mbox{at}}}}
\newcommand{\pcafter}{\textcolor[HTML]{3c6382}{\textsf{\mbox{after}}}}
\newcommand{\pcescape}{\textcolor[HTML]{3c6382}{\textsf{\mbox{escape}}}}
\newcommand{\pcbreakto}{\textcolor[HTML]{3c6382}{\textsf{\mbox{break-to}}}}
\newcommand{\pcbreaksof}{\textcolor[HTML]{3c6382}{\textsf{\mbox{breaks-of}}}}
\newcommand{\pcin}{\textcolor[HTML]{3c6382}{\textsf{\mbox{in}}}}
\newcommand{\pclabs}{\textcolor[HTML]{3c6382}{\textsf{\mbox{labs}}}}
\newcommand{\pclabx}{\textcolor[HTML]{3c6382}{\textsf{\mbox{labx}}}}
\newcommand{\definiendum}{D}
\newcommand{\universe}{\mathbbm{U}}
\newcommand{\increasing}{\xrightarrow{\nearrow}}
\newcommand{\tojoin}{\xrightarrow{\sqcup}}
\newcommand{\tomeet}{\xrightarrow{\sqcap}}
\renewcommand{\lambda}{\textcolor[HTML]{8e44ad}{\lambda}}
\newcommand{\abs}{\mathscr{A}}
\newcommand{\crc}{\mathscr{C}}
\newcommand{\preproperty}{ℙ}
\newcommand{\postproperty}{ℚ}
\newcommand{\post}{\mbox{post}}
\newcommand{\dpost}{\widetilde{\mbox{post}}}
\newcommand{\lfp}{{\kern-0.20em}\mathbin{\mbox{lfp}}}
%% End book symbols

\usepackage{tikz}
\usepackage{fancyvrb}
\usepackage[T1]{fontenc}

\begin{document}

\maketitle

Progress on pegreg stalled because of concerns about the correctness of the algorithm that created an FST for possessive star.

Nonetheless, I believe it is still possible to define possessive star in finite state machines:

Parsing expression grammars are usually defined with respect to their input string, in effect embedding a ``continuation'' string into the semantics of a parsing expression grammar.

In the paper ``Towards Typed Semantics for Parsing Expression Grammars'' (2019), Rebeiro et. al. present an operational semantics for PEG that includes a left-recursive version of the
star operator. I believe this semantics can be encoded as a fixpoint, viewing the strings ``to be matched'' as the set of all strings, then shown equivalent to the following rules:

\begin{center}
$
  \begin{array}{l l}
    R_p \llbracket \langle \mbox{ch}, \epsilon \rangle \rrbracket = \{ \mbox{ch} \} & \mbox{character literal} \\
    R_p \llbracket \langle \mbox{ch}, e_2 \rangle \rrbracket = \{ \mbox{ch}.s \mid s \in R_p \llbracket \langle e_2, \epsilon \rangle \rrbracket \} & \mbox{character literal with continuation} \\
    R_p \llbracket \langle e_1.e_2, e_3 \rangle \rrbracket = R_p \llbracket \langle e_1, e_2.e_3 \rangle \rrbracket & \mbox{concatenation} \\
    R_p \llbracket \langle e_1/e_2, e_3 \rangle \rrbracket = \{ s_1s_3, s_2s_3 \mid s_1 \in R_p \llbracket \langle e_1, \epsilon \rangle \rrbracket, s_3 \in R_p \llbracket \langle e_3, \epsilon \rangle \rrbracket, s_2 \in R_p \llbracket \langle e_2, \epsilon \rangle \rrbracket \setminus R_p \llbracket \langle e_1, \epsilon \rangle \rrbracket \} & \mbox{ordered choice} \\
    F_p \llbracket \langle e_1\ast, \epsilon \rangle \rrbracket(X) = \{s_1 \mid s_1 \in R_p \llbracket \langle e_1, \epsilon \rangle \rrbracket \} \cup \{s_1s_2 \mid s_1 \in X \land s_2 \in X \} & \mbox{greedy repitition} \\
    R_p \llbracket \langle e_1\ast, \epsilon\rangle \rrbracket = \lfp F_p & \ast \epsilon\mbox{-case} \\
    R_p \llbracket \langle e_1\ast, e_2 \rangle \rrbracket = \{s_1s_2 \mid s_1 \in R_p \llbracket \langle e_1\ast, \epsilon \rangle \rrbracket, s_2 \in R_p \llbracket e_2 \rrbracket \setminus R_p \llbracket e_1 \rrbracket \} & \ast \mbox{general case} \\
  \end{array}
$
\end{center}

The set minus operation can then be encoded in a finite state machine: $L_1 \setminus L_2 = L_1 \cap \neg(L_2) = \neg(L_1 \cup \neg(L_2))$.

\end{document}
